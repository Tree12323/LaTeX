\documentclass{article}

\usepackage{ctex}
\usepackage{amsmath}

\title{初步公式}
\author{Yong}
\date{\today}

\begin{document}
	\maketitle
	\tableofcontents
	\section{简介}
	 \LaTeX{}将排版内容分为文本模式和数学模式。文本模式用于普通文本排版,数学模式用于数学公式排版。
	\section{行内公式}
	\subsection{美元符号}
	交换律是$a+b=b+a$,如$1+2=2+1$。
	\subsection{小括号}
	交换律是\(a+b=b+a\),如\(1+2=2+1\)。
	\subsection{math环境}
	交换律是 \begin{math}a+b=b+a\end{math},如 \begin{math}1+2=2+1\end{math}。
	\section{上下标}
	\subsection{上标}
	$3x^{20} - x + 2 = 0$
	
	$3x^{3x^{20} - x + 2 = 0} - x + 2 = 0$
	\subsection{下标}
	$a_0, a_1, a_2 $
	
	$a_0, a_1, a_2, ..., a_{3x^{20} - x + 2 = 0} $
	\section{希腊字母}
	$\alpha$
	$\beta$
	$\gamma$
	$\epsilon$
	$\pi$
	$\omega$
	
	$\Gamma$
	$\Delta$
	$\Theta$
	$\Pi$
	$\Omega$
	
	$\alpha^3 + \beta^2 + \gamma = 0$
	\section{数学函数}
	$\log$
	$\sin$
	$\cos$
	$\arcsin$
	$\arccos$
	$\ln$
	
	$\sin^2 x + \cos^2 x = 1$
	
	$y = \arcsin x$
	
	$y= \sin^{-1} x$
	
	$y = \log_2 x$
	
	$y = \ln x$
	
	$\sqrt{2}$
	$\sqrt{x^2 + y^2}$
	$\sqrt{2 + \sqrt{2}}$
	$\sqrt[4]{x}$
	\section{分式}
	大约体积是$3/4$
	大约体积是$\frac{3}{4}$
	
	$\frac{x}{x^2 + x + 1}$
	
	$\frac{\sqrt{x - 1}}{\sqrt{x + 1}}$
	
	$\frac{1}{1 + \frac{1}{x}}$
	
	$\sqrt{\frac{x}{x^2 + x + 1}}$
	
	\section{行间公式}
	\subsection{美元符号}
	交换律是
	$$a+b=b+a$$
	如 $$1+2=2+1$$。
	\subsection{中括号}
	交换律是
	\[a+b=b+a\]
	如 \[1+2=2+1\]。
	\subsection{displaymath 环境}
	交换律是 \begin{displaymath}a+b=b+a\end{displaymath}
	如 \begin{displaymath}1+2=2+1\end{displaymath}。
	\subsection{自动编号公式equation环境}
	交换律见式 \ref{eq:commuative}
	\begin{equation}
		a+b=b+a \label{eq:commuative}
	\end{equation}
	\subsection{不自动编号公式equation*环境}
	\begin{equation*}
		a+b=b+a 
	\end{equation*}

	公式的编号与交又引用也是自动实现的,大家在排版中,要习惯于采用自动化的方式处理诸如图、表、公式的编号与交叉引用。再如公式 \ref{eq:pol}
	\begin{equation}
		x^5 + 7x^3 + 4x = 0 \label{eq:pol}
	\end{equation}

\end{document}