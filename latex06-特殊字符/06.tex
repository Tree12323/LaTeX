\documentclass{article}

\usepackage{ctex}

\title{特殊符号}
\author{勇}
\date{\today}

\begin{document}
	\maketitle
	\section{空白字符}
	% 空行分段,多个空行等同1个
	% 自动缩进,绝对不能使用空格代替
	% 英文中多个空格处理为1个空格,中文中空格将被忽略
	% 汉字与其它字符的间距会自动由 XeLateX处理
	% 禁止使用中文全角空格
	While previous research has looked into how whole body vibration (WBV) functions as an 'exercise mimetic', there's still a lot we don't understand about how vibration alone can produce such significant changes in the body.
	
	虽然之前的研究调查了全身振动(WBV)是如何作为一个“模拟运动”作用的,但我们对于仅振动是如何在身体上产生如此巨大的变化还很不了解。
	
	a \quad b
	
	a \qquad b
	
	\section{\LaTeX 控制符}
	
	\# \$ \% \{ \} \~{} \_{} \^{} \textbackslash  \&
	
	\section{排版符号}
	
	\S \P \dag \ddag \copyright \pounds
	
	\section{\TeX 标志符号}
	
	\TeX{} \LaTeX{} \LaTeXe{}
	
	\section{引号}
	
	` '
	
	`` ''
	
	``你好''
	
	\section{连字符}
	
	- -- ---
	
	\section{非英文字符}
	
	\oe \OE \ae \AE \aa \AA \o \O \l \L \ss \SS !` ?`
	
	\section{重音符号}
	
	\`o \'o \^o \''o \~o \=o \.o \u{o} \v{o} \H{o} \r{o} \t{o} \b{o} \c{o} \d{o}
	
\end{document}